\testCom
{%Номер задачи
	3.164
}
{%Условие
	Катушка с индуктивностью L = 0,70 Гн и r = 20 Ом соединена последовательно с безындукционным сопротивлением R, и между концами этой цепи приложено переменное напряжение с действующим значением U = 220 В и частотой $\omega = 314 с^{-1}$. При каком значении  сопротивления R в цепи будет выделяться максмальная тепловая мощность. Чему она равна.
}
{%Дано
	$L, r, U, \omega$
}
{%Найти
	R- ? 
	${P}_{max} $-?
}
{%Решение
	$z=R+r+\omega L$ комплексное сопротивление цепи\\
	$\abs{z}=\sqrt{(R+r)^2 + (\omega L)^2}$\\
	${I}_{0} = \frac{{U}_{0}}{\abs{z}}$\\
	$P(R)=\frac{I^2(R+r)}{2}=\frac{U^2(R + r)}{2((R + r)^2 + (\omega L)^2)}$ амплитудная мощность\\
	$\frac{d}{dR}P(R)=-U^2(R+r -\omega L)$\\
	$\frac{R+r+\omega L}{(R^2+ 2Rr+r^2+\omega^2 L^2)^2}$\\
	${R}_{max}=\omega L - r$\\
	${P}_{max}=P({R}_{max})$\\
	${P}_{max}= \frac{U^2(\omega L)}{(\omega L)^2 + (\omega L)^2}$\\
	${P}_{max}=\frac{U^2}{2(\omega L)}$\\
}

